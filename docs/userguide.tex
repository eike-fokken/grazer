\documentclass[a4paper]{article}

\usepackage{todonotes}
\usepackage{booktabs}
\usepackage{hyperref}
\usepackage{listings}
\usepackage{xcolor}
\usepackage{tabularx}


\definecolor{eclipseStrings}{RGB}{42,0.0,255}
\definecolor{eclipseKeywords}{RGB}{127,0,85}
\colorlet{numb}{magenta!60!black}

\lstdefinelanguage{json}{
    basicstyle=\normalfont\ttfamily,
    commentstyle=\color{eclipseStrings}, % style of comment
    stringstyle=\color{eclipseKeywords}, % style of strings
    numbers=left,
    numberstyle=\scriptsize,
    stepnumber=1,
    numbersep=8pt,
    showstringspaces=false,
    breaklines=true,
    frame=lines,
    backgroundcolor=\color{white}, %only if you like
    string=[s]{"}{"},
    comment=[l]{:\ "},
    morecomment=[l]{:"},
    literate=
        *{0}{{{\color{numb}0}}}{1}
         {1}{{{\color{numb}1}}}{1}
         {2}{{{\color{numb}2}}}{1}
         {3}{{{\color{numb}3}}}{1}
         {4}{{{\color{numb}4}}}{1}
         {5}{{{\color{numb}5}}}{1}
         {6}{{{\color{numb}6}}}{1}
         {7}{{{\color{numb}7}}}{1}
         {8}{{{\color{numb}8}}}{1}
         {9}{{{\color{numb}9}}}{1}
}

\newcommand{\sco
}{\textunderscore{}}


\begin{document}


\todo[inline]{Note that this document is a work in progress and not yet very helpful. This will hopefully and probably change in July 21.}
\section{Grazer}
\label{sec:grazer}
This is the user guide for grazer.

\subsection{Usage}
\label{sec:usage}

\subsubsection{Input files}
\label{sec:input-files}
Grazer is controlled by input files, which contain json code.
The most important such file is called \emph{problem\sco data.json}. Its top-level structure is given in \autoref{fig:problem_data.json}.

\begin{figure}[ht]
  \centering
  \begin{lstlisting}[language=json,firstnumber=1]
    {
      "time_evolution_data": { ... },
      "initial_values": { ... },
      "problem_data": { ... }
    }
  \end{lstlisting}
  \caption{problem\sco data.json}
  \label{fig:problem_data.json}
\end{figure}
Here \verb|"time_evolution_data"| provides all data on time and newton solver. An example is given in

\begin{figure}[ht]
  \centering
\begin{lstlisting}[language=json,firstnumber=1]
    "time_evolution_data":
    {
        "use_simplified_newton": true,
        "maximal_number_of_newton_iterations": 100,
        "tolerance": 1e-8,
        "retries": 5,
        "start_time": 0,
        "end_time": 86400,
        "desired_delta_t": 1800
    }
\end{lstlisting}
  \caption{time evolution data entry}
  \label{fig:time_evolution_data}
\end{figure}

\begin{table}[ht]
  \centering
  \begin{tabularx}{\textwidth}{llXl}
    \toprule
    name (abr.) & type & explanation &example \\
   \midrule
    u.\sco s.\sco n.&Boolian& Choose to update Jacobian on every iteration(false) or only at the beginning of a time step(true)& true \\
    m.\sco n.\sco of\sco n.\sco i. &Integer& Abort if this many iterations didn't converge & 50\\
    tolerance&Float& Declare time step converged, if the objective function value is this close to zero& 1e-8\\
p    retries&Integer& Only useful for stochastic inputs: Retry the step, if it didn't converge with new stochastic sample & 0\\
    start\sco time&Float& Start time of the simulation in seconds& 0\\
    end\sco time&Float& End time of the simulation in seconds& 3600\\
    desired\sco delta\sco t&Float& Given in seconds. The next-smaller number
    that is a divisor of endtime-starttime is chosen as timestep & 60\\
    \bottomrule
  \end{tabularx}
  \caption{All keys in time evolution data}
  \label{tab:time_evolution_data}
\end{table}

The internal structure of the intial\sco values json is given in \autoref{fig:initvalues}.
The structure at the moment is not very interesting, as it only holds a file name to an additional initialvalues json which unsurprisingly holds initial values.
This is sorted by subproblem. When (and if) support for different subproblem types arrives, this structure will probably change.

\begin{figure}[ht]
  \centering
\begin{lstlisting}[language=json,firstnumber=1]
  "initial_values":{
    "subproblems":
    {
      "Network_problem":
      {
        "initial_json":"initial.json"
      }
    }
  }
\end{lstlisting}
  \caption{intialvalues json}
  \label{fig:initvalues}
\end{figure}

Lastly the problem data can contain defaults. These defaults are applied when
properties in the \emph{topology json} are missing. Since the boundary json and
control json is merged into the topology json at runtime it is also possible to
set defaults for these, using the keys boundary\sco values and control\sco
values respectively. Since defaults are applied before validating the input,
providing defaults for \emph{required} properties allows you to omit them in the
actual input. An example problem\sco data json is given in Figure~\ref{fig:problem_data}

\todo{
  If you rely on Schemas generated by grazer schema to validate your input
  schemas you need to regenerate them after modifying defaults to make sure
  the required property is removed when defaults exist.
  % \begin{itemize}
    % \item
    You could use
    \href{https://github.com/inotify-tools/inotify-tools/wiki}{inotifywait} to
    trigger grazer schema automatically whenever the problem\sco data.json is
    modified.
    % \item
    You could simply ignore
    these warnings as the static schema files are only for your own
    convenience and grazer uses on-the-fly generated schemas internally.
  % \end{itemize}
}

\begin{figure}[ht]
  \centering
\begin{lstlisting}[language=json,firstnumber=1]
  "problem_data": {
      "subproblems": {
          "Network_problem": {
              "defaults": {
                  "StochasticPQnode": {
                      "number_of_stochastic_steps": 1000,
                      "boundary_values": {
                          "data": [
                          {"time": 0, "values": [0, 1]}
                          ]
                      },
                      "control_values": {
                          "data": [
                          {"time": 0, "values": [0]}
                          ]
                      }
                  },
                  "Pipe": {
                      "balancelaw": "Isothermaleulerequation",
                      "scheme": "Implicitboxscheme",
                      "desired_delta_x": 10000
                  }
              }
          }
      }
  }
\end{lstlisting}
  \caption{problem\sco data json}
  \label{fig:problem_data}
\end{figure}

\subsubsection{Internal structure}
\label{sec:internal-structure}
After all filenames have been replaced with their file contents, the final json objects are handed to the objects, that finally carry out the computations.

A Problem object passes on subproblem jsons to the factory functions of subproblem objects (at the moment only Networkproblem is implemented).
The Networkproblem needs a network.
This is constructed from the arrays of object jsons. First the nodes, then the edges. The network is then passed to the Networkproblem object.


After construction the main loop starts.
The timeevolver object loads the initial state from the initial condition json and runs the time steps.
Therefore it calls \emph{evaluate} on the problem object, which in turn calls \emph{evaluate} on its subproblems.
These handle the evaluation. The Networkproblem also hands \emph{evaluate} to its members, namely the nodes and edges.

\subsubsection*{evaluate}

\subsection{Extending Networkproblem}
\label{sec:extend-netw}
The easiest way to extend grazer for your needs, is to create new classes that inherit from \emph{Node} or \emph{Edge} and
(most of these at least do) from either \emph{Statecomponent} (if the new componenttype has its own variables) or
\emph{Equationcomponent} (if the new component sets model equations in other components.)

Examples are \emph{Transmissionline}, which only inherits from \emph{Edge}, because it holds some parameters but neither sets equations, nor holds variables.
On the other hand \emph{Pipe} inherits from \emph{Edge} and \emph{Statecomponent}, because it has its own variables, namely pressure and flow values and sets their equations.
Lastly the gas nodes \emph{Source, Sink} and \emph{Innode} inherit from \emph{Node} and \emph{Equationcomponent}, because they set equations, but do so in the gas edges, that meet in the node. These nodes have no variables of their own.


\end{document}
