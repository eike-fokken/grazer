\documentclass[a4paper]{article}

\usepackage{todonotes}

\usepackage{hyperref}
\usepackage{listings}
\usepackage{xcolor}


\usepackage{tabularx}
\usepackage{ltablex}
\usepackage{booktabs}

\definecolor{eclipseStrings}{RGB}{42,0.0,255}
\definecolor{eclipseKeywords}{RGB}{127,0,85}
\colorlet{numb}{magenta!60!black}

\lstdefinelanguage{json}{
    basicstyle=\normalfont\ttfamily,
    commentstyle=\color{eclipseStrings}, % style of comment
    stringstyle=\color{eclipseKeywords}, % style of strings
    numbers=left,
    numberstyle=\scriptsize,
    stepnumber=1,
    numbersep=8pt,
    showstringspaces=false,
    breaklines=true,
    frame=lines,
    backgroundcolor=\color{white}, %only if you like
    string=[s]{"}{"},
    comment=[l]{:\ "},
    morecomment=[l]{:"},
    literate=
        *{0}{{{\color{numb}0}}}{1}
         {1}{{{\color{numb}1}}}{1}
         {2}{{{\color{numb}2}}}{1}
         {3}{{{\color{numb}3}}}{1}
         {4}{{{\color{numb}4}}}{1}
         {5}{{{\color{numb}5}}}{1}
         {6}{{{\color{numb}6}}}{1}
         {7}{{{\color{numb}7}}}{1}
         {8}{{{\color{numb}8}}}{1}
         {9}{{{\color{numb}9}}}{1}
}

\newcommand{\sco
}{\textunderscore{}}


\begin{document}


\section{grazer}
\label{sec:grazer}
This is the user guide for grazer.

\subsection{Installation}
\label{sec:installation}
In order to install grazer you need at least CMake version 3.9 and a working C++ compiler.
Supported and tested are gcc 9 or later, clang 9 or later and MSVC 2019 or later.
For building the documentation you need in addition doxygen, graphviz and pdflatex with all packages in the source of this Latex-document.

Having all this, grazer can be downloaded, built and installed with
\begin{lstlisting}[language=bash,caption={Build},basicstyle=\scriptsize\ttfamily\color{blue}]
git clone https://github.com/eike-fokken/grazer.git
cd grazer
git submodule update --init --recursive --depth=1
cmake -DCMAKE_BUILD_TYPE=Release -S . -B release
cmake --build release
\end{lstlisting}
Afterwards there is a grazer binary in \verb|.../grazer/release/src/Grazer| called \verb|grazer| or \verb|grazer.exe| (on Windows).

You can put this binary into your path (so that you can just write \verb|grazer| instead of specifying its own location) with
\begin{lstlisting}[language=bash,caption={Installation},basicstyle=\scriptsize\ttfamily\color{blue}]
cmake --build release --target install
\end{lstlisting}

For building the docs you can call
\begin{lstlisting}[language=bash,caption={Building docs},basicstyle=\scriptsize\ttfamily\color{blue}]
cmake --build release --target docs
\end{lstlisting}
If you have pdflatex, there should now be this Latex-document in\\
\verb|.../grazer/release/docs/userguide.pdf|.
If you have doxygen, there should also be a technical documentation, whose homepage is\\
\verb|.../grazer/release/docs/html/index.html|.

If you want to build a debug version of grazer, you have to replace \emph{release} with \emph{debug} and \emph{Release} with \emph{Debug} in all commands above.
Because of the low performance of the debug version, you should not install this version (i.e. don't run \verb|cmake --build debug --target install|) and instead call it explicitly by \verb|.../grazer/debug/Grazer/grazer| for debugging purposes.

\subsection{Usage}
\label{sec:usage}
The simulation capabilities of grazer are invoked by calling
\begin{lstlisting}[language=bash,caption={Invocation},basicstyle=\scriptsize\ttfamily\color{blue}]
grazer run <path/to/problemfolder>
\end{lstlisting}
Grazer should tell you of any errors it encountered. If no errors are reported, an outputfile\\
\verb|problemfolder/output/outputXXXXXXX.json| should appear. Here
\emph{XXXXXXX} stands for some number which makes different outputfiles unique.
This number is the number of milliseconds since the UNIX epoch (01.01.1970).
Therefore output files are ordered by time of creation.


\subsubsection{Input files}
\label{sec:input-files}
grazer is controlled by input files, which contain json code.
The most important such file is called \emph{problem\sco data.json}. Its top-level structure is given in \autoref{fig:problem_data.json}.

\begin{figure}[ht]
  \centering
  \begin{lstlisting}[language=json,firstnumber=1]
    {
      "time_evolution_data": { ... },
      "initial_values": { ... },
      "problem_data": { ... }
    }
  \end{lstlisting}
  \caption{problem\sco data.json}
  \label{fig:problem_data.json}
\end{figure}
Here \verb|"time_evolution_data"| provides all data on time and newton solver. An example is given in

\begin{figure}[ht]
  \centering
\begin{lstlisting}[language=json,firstnumber=1]
    "time_evolution_data":
    {
        "use_simplified_newton": true,
        "maximal_number_of_newton_iterations": 100,
        "tolerance": 1e-8,
        "retries": 5,
        "start_time": 0,
        "end_time": 86400,
        "desired_delta_t": 1800
    }
\end{lstlisting}
  \caption{time evolution data entry}
  \label{fig:time_evolution_data}
\end{figure}

\begin{table}[ht]
  \centering
  \begin{tabularx}{\textwidth}{llXl}
    \toprule
    name (abr.) & type & explanation &example \\
   \midrule
    u.\sco s.\sco n.&Boolian& Choose to update Jacobian on every iteration(false) or only at the beginning of a time step(true)& true \\
    m.\sco n.\sco of\sco n.\sco i. &Integer& Abort if this many iterations didn't converge & 50\\
    tolerance&Float& Declare time step converged, if the objective function value is this close to zero& 1e-8\\
p    retries&Integer& Only useful for stochastic inputs: Retry the step, if it didn't converge with new stochastic sample & 0\\
    start\sco time&Float& Start time of the simulation in seconds& 0\\
    end\sco time&Float& End time of the simulation in seconds& 3600\\
    desired\sco delta\sco t&Float& Given in seconds. The next-smaller number
    that is a divisor of endtime-starttime is chosen as timestep & 60\\
    \bottomrule
  \end{tabularx}
  \caption{All keys in time evolution data}
  \label{tab:time_evolution_data}
\end{table}

The internal structure of the intial\sco values json is given in \autoref{fig:initvalues}.
The structure at the moment is not very interesting, as it only holds a file name to an additional initialvalues json which unsurprisingly holds initial values.
This is sorted by subproblem. When (and if) support for different subproblem types arrives, this structure will probably change.

\begin{figure}[ht]
  \centering
\begin{lstlisting}[language=json,firstnumber=1]
  "initial_values":{
    "subproblems":
    {
      "Network_problem":
      {
        "initial_json":"initial.json"
      }
    }
  }
\end{lstlisting}
  \caption{intialvalues json}
  \label{fig:initvalues}
\end{figure}

There is also a boundary.json file which contains boundary values which is similar to initial.json.

\subsubsection{topology.json}
\label{sec:topology.json}
The most important file is topology.json. In it the underlying graph of the
problem and all components in it are defined. Its structure is very similar to
GasLib xml files and split into connections and nodes of the graph. At the
moment the easiest way to understand its structure is to have a look at the
example files in the data directory. The problem case \emph{one\sco pipeline}
consists of a single gas pipeline between a gas source and a gas sink, the
problem case \emph{base} consists of the GasLib-134 net connected to the
ieee-300 bus system via gas powerplants and power-to-gas stations and is a
bigger example containing almost all components present in grazer.


Lastly the problem data can contain defaults. These defaults are applied when
properties in the \emph{topology json} are missing. Since the boundary json and
control json is merged into the topology json at runtime it is also possible to
set defaults for these, using the keys boundary\sco values and control\sco
values respectively. Since defaults are applied before validating the input,
providing defaults for \emph{required} properties allows you to omit them in the
actual input. An example problem\sco data json is given in Figure~\ref{fig:problem_data}

\todo[inline,color=yellow]{\emph{Note:}\\
  If you rely on Schemas generated by grazer schema to validate your input
  jsons you need to regenerate the schemas after modifying defaults to make sure
  the defaulted property is not anymore required in the topology file.
  % You could use
  % \href{https://github.com/inotify-tools/inotify-tools/wiki}{inotifywait} to
  % trigger grazer schema automatically whenever the problem\sco data.json is
  % modified.
  % % \item
}
You could also simply ignore
these warnings as the static schema files are only for your own
convenience and grazer uses on-the-fly generated schemas internally.
\begin{figure}[ht]
  \centering
\begin{lstlisting}[language=json,firstnumber=1]
  "problem_data": {
      "subproblems": {
          "Network_problem": {
              "defaults": {
                  "StochasticPQnode": {
                      "number_of_stochastic_steps": 1000,
                      "boundary_values": {
                          "data": [
                          {"time": 0, "values": [0, 1]}
                          ]
                      },
                      "control_values": {
                          "data": [
                          {"time": 0, "values": [0]}
                          ]
                      }
                  },
                  "Pipe": {
                      "balancelaw": "Isothermaleulerequation",
                      "scheme": "Implicitboxscheme",
                      "desired_delta_x": 10000
                  }
              }
          }
      }
  }
\end{lstlisting}
  \caption{problem\sco data json}
  \label{fig:problem_data}
\end{figure}

\subsubsection{Internal structure}
\label{sec:internal-structure}
\todo[inline]{This section is not yet ready, please raise issues on github about it.}
After all filenames have been replaced with their file contents, the final json objects are handed to the objects, that finally carry out the computations.

A Problem object passes on subproblem jsons to the factory functions of subproblem objects (at the moment only Networkproblem is implemented).
The Networkproblem needs a network.
This is constructed from the arrays of object jsons. First the nodes, then the edges. The network is then passed to the Networkproblem object.


After construction the main loop starts.
The timeevolver object loads the initial state from the initial condition json and runs the time steps.
Therefore it calls \emph{evaluate} on the problem object, which in turn calls \emph{evaluate} on its subproblems.
These handle the evaluation. The Networkproblem also hands \emph{evaluate} to its members, namely the nodes and edges.

\subsubsection{Helper functions}%
\label{sec:helper-functions}
% \begin{tabularx}{\linewidth}{ c X }
%   \caption{Changes Required from Examiner A}\\\toprule\endfirsthead
%   \toprule\endhead
%   \midrule\multicolumn{2}{r}{\itshape continues on next page}\\\midrule\endfoot
%   \bottomrule\endlastfoot
%   \textbf{page} & \textbf{Content} \\\midrule
{  \centering
  \begin{tabularx}{\textwidth}{lX}
    \toprule
    name (-.cpp) & explanation \\
   \midrule
    change\sco inflow & Linearly scales the first two flux inputs in a boundary values file to sum to the given value. Only makes sense for constant influx \\
    compute\sco reference\sco discrepance & Computes maximal discrepance between a given reference output of a grazer problem and a collection of grazer outputs for the same topology \\
    construct\sco gaspowerconnection & A function to move a component from one type to another \\
    draw\sco gaslib\sco graph & A function to write out a latex file which draws a graph of the gaslib part of a topology file \\
    draw\sco heatmap\sco gaslib\sco graph & A function to write out a latex file which draws a heatmap of a gas net \\
    draw\sco heatmap\sco power & A function to write out a latex file which draws a heatmap of a power net, where the color shows maximal power attained over a simulation time frame \\
    draw\sco power\sco graph & A function to write out a latex file which draws a graph of the power part of a topology file \\
    extract\sco new\sco initial\sco condition & Writes out the state at the given time passed in an results file into a file that can be used as an initial value file \\
    generate\sco printing\sco csv & Outputs data from the output.json of grazer to a csv file \\
    get\sco mean\sco sigma & Computes mean and standard deviation of a collection of grazer outputs \\
    get\sco quantiles & Computes quantiles for a collection of grazer outputs \\
    include\sco seed\sco in\sco boundaryvalues & Writes the seeds used in an output file into the boundary values of a problem folder \\
    increase\sco pressure & Adds a pressure value on all existing pressure values in the initial value file in the problem folder \\
    insert\sco positions & Copies over positions from a topology json file \\
    move\sco to\sco category & A function to move a component from one type to another \\
    normalize\sco jsons & Normalizes json files for a higher readability \\
    oup & Simulates an Ornstein-Uhlenbeck-process with the Euler Maruyama method \\
    pv\sco to\sco vphi & A function to move a component from one type to another \\
    remove\sco seed\sco from\sco boundaryvalues & Removes all seed entries from boundary.json file \\
    stringify\sco value\sco json & Converts all the values in each datapoint into strings \\
    sum\sco outflow & Prints the summed outflux of a network \\
    \bottomrule\\
    \caption{All helper functions. For a more detailed description use the technical documentation.}
  \end{tabularx}
}
  
  % \label{tab:helper_functions}


\subsubsection*{evaluate}

\subsection{Extending Networkproblem}
\label{sec:extend-netw}
\todo[inline]{This section is not yet ready, please raise issues on github about it.}
The easiest way to extend grazer for your needs, is to create new classes that inherit from \emph{Node} or \emph{Edge} and
(most of these at least do) from either \emph{Statecomponent} (if the new componenttype has its own variables) or
\emph{Equationcomponent} (if the new component sets model equations in other components.)

Examples are \emph{Transmissionline}, which only inherits from \emph{Edge}, because it holds some parameters but neither sets equations, nor holds variables.
On the other hand \emph{Pipe} inherits from \emph{Edge} and \emph{Statecomponent}, because it has its own variables, namely pressure and flow values and sets their equations.
Lastly the gas nodes \emph{Source, Sink} and \emph{Innode} inherit from \emph{Node} and \emph{Equationcomponent}, because they set equations, but do so in the gas edges, that meet in the node. These nodes have no variables of their own.


\end{document}
